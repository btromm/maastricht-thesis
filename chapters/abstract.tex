\begin{abstract}
Serotonergic psychedelic compounds are gaining an increasing amount of scientific and medical interest for their
therapeutic potential and unique effects on consciousness, prompting researchers to consider them as
useful tools for probing the neural correlates of consciousness and functional dynamics and organization of the brain.
Previous work has revealed that these compounds can significantly alter the functional dynamics and organization of the brain.
While substances like cannabis are well-characterized with regard to their chronic effects, a dearth of research has thus far failed
to establish alterations in functional dynamics and brain hierarchy following long-term use of psychedelics. A recent proposal has argued that psychedelics act by relaxing high-level priors and flattening the hierarchical organization of the brain. Here, we merge recent work quantifying the hierarchical relationship between the arrow of time
and brain dynamics with whole-brain modeling to examine the trophic coherence, a measure of hierarchy, of the brain
under ayahuasca, DMT, and cannabis in chronic and occasional users. We recapitulate that psychedelics decrease
irreversibility and establish that psychedelics in both long-term users and naive users flatten the brain's
hierarchical organization and significantly reconfigure the regional and network-level hierarchical regime, an
effect distinct from cannabis in both chronic and occasional users.
\end{abstract}