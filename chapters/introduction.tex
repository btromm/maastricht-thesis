
\chapter{Introduction}
    Psychedelics, including drugs such as psilocybin (4-PO-DMT), lysergic
acid diethylamide (LSD), and dimethyltryptamine (DMT), an ingredient in
the ritual brew ayahuasca, have seen a massive resurgence in medical and
scientific interest in the last decade. These compounds induce
significant altered states of consciousness, characterized by vivid
imagery, the sense of being in another reality, and dissolution or
loosening of the ego, or one's sense of self. Previous research has
shown psychedelics to have lasting impacts on personality and mental
health outcomes \parencite{Barbosa2009,Griffiths2008}.
Personality traits are considered to be rigid and inflexible in healthy
adults, and despite this psychedelics have been shown to increase trait
openness several months after an acute experience \parencite{Griffiths2011,MacLean2011, McCrae1997}. Evidence is
mounting that psychedelics can improve outcomes for a variety of
conditions, including depression \parencite{Raison2023}, OCD \parencite{Moreno2006}, smoking addiction \parencite{Johnson2017}, and alcohol
use disorder \parencite{Bogenschutz2015}. The strongest evidence for the
therapeutic use of psychedelics lies with depressive and anxiety
symptoms associated with life-threatening illnesses, such as cancer \parencite{Griffiths2008,Griffiths2006,Grob2011}.
These results highlight the need for further research into their effects
on the brain.

The propensity of psychedelics to induce significant altered states also
makes them an interesting candidate to assist in understanding normal
and abnormal brain function and identifying the neural correlates of
consciousness. Previous work has showed that psychedelics modulate
emotion \parencite{Roseman2019}, sense of self \parencite{Lebedev2015}, perceptual processing \parencite{Kometer2016}, and
increase feelings of connectedness \parencite{Carhart-Harris2018b}. A large body
of evidence indicates that psychedelics trigger these effects via 5-HT2a
agonism \parencite{Nichols2016}, which are localized in the cortex, primarily in
the association cortices \parencite{Nichols2016,Weber2010}. It is not
well-known what the downstream effects of 5-HT2a agonism are with regard
to global brain function, though the rate of research has been growing
steadily. Recent work has shown that the therapeutic effects of
psychedelics, in rodents, are mediated by TrkB agonism \parencite{Moliner2023}. A convergent body of research has indicated that psychedelics
increase measures of brain complexity during the acute psychedelic
experience \parencite{Lebedev2016,Tagliazucchi2014,Viol2017}. These results build support for the theory that, during
normal waking consciousness, the human brain operates at
sub-criticality, or perfect criticality \parencite{Carhart-Harris2014}.
Criticality has been shown to play a role in a variety of processes in
nature, and in the brain it has been speculated to play a role in
neuronal avalanches \parencite{Beggs2003}. Early studies indicate that
psychedelics suppress the default mode network (DMN), decreasing
within-network integrity and segregation from other resting-state
networks \parencite{Buckner2008,Carhart-Harris2012,Johnson2019,Muller2018,Petri2014,Roseman2014,Tagliazucchi2014,Timmermann2023}. Furthermore,
decoupling between the DMN and medial temporal lobes has been shown
\parencite{Carhart-Harris2014}, which has led to the suggestion that the
effects of psychedelics are mediated by suppression and desegregation of
networks in the association cortex \parencite{Girn2022}.

The evaluation of long-term outcomes of psychedelic use, as well as the
development of an understanding of how psychedelics modulate brain
dynamics long-term, has not been extensively demonstrated. \textcite{Bouso2015} found significant differences in cortical thickness in midline
structures of the brain, with thinning in the middle frontal gyrus,
inferior frontal gyrus, precuneus, superior frontal gyrus, superior
occipital gyrus, and posterior cingulate cortex (PCC), a key node of the
default mode network (DMN). Thickening was found in
the precentral gyrus and anterior cingulate cortex. Importantly, the
medial prefrontal cortex (mPFC), ACC, and PCC have been associated with
the acute effects of psychedelics \parencite{Riba2006}, and disruption of
the default mode network is a key, convergent finding in psychedelic
neuroimaging studies \parencite{Carhart-Harris2017a,McCulloch2022}. Furthermore, previous work has demonstrated a significant
increase in resting-state functional connections across the brain one
month after psilocybin use \parencite{Barrett2020}.

It's believed that psychedelics shift brain dynamics into a more
flexible, diverse, and sensitive mode that is more tuned for information
sharing and propagation \parencite{Atasoy2017,Carhart-Harris2014, Carhart-Harris2017, Daws2022, Girn2022, Girn2023, Lord2019, Singleton2022, Tagliazucchi2014, Tagliazucchi2016, Timmermann2023} The state of unconstrained cognition induced by psychedelics has been previously conceptualized as a
flattening of the attractor landscape -- the brain
has a reduced tendency to exhibit metastability within potentially
maladaptive local minima states, and is thus able to more easily move
between metastable local minima \parencite{Carhart-Harris2007, Carhart-Harris2014, Kraehenmann2017, Kraehenmann2017a, Girn2023, Daws2022, Singleton2022}. This is consistent with the REBUS (`RElaxed Beliefs Under
Psychedelics') model of psychedelic action as proposed by \textcite{Carhart-Harris2019a}, which proposes that psychedelics increase
entropy in the brain, resulting in `critical primary states' which allow
for the relaxation of priors and decreased rigidity of thinking and
cognition. Entropy is an information-theoretic measure derived from
thermodynamics which indexes the fundamental temporal complexity or
diversity of a trajectory of the dynamics of a system and the
unpredictability of the trajectory, specifically the
relative frequency of values that a signal in a system takes on \parencite{Girn2023}. It is
often characterized more simply as disorder or randomness in a system.
Similarly, it is thought that psychedelics flatten the global functional
hierarchy of the brain by increasing crosstalk between hierarchical
extremes \parencite{Girn2022, Timmermann2023}. Hierarchy here is
defined as the directional asymmetry of information flow throughout the
brain, or alternatively, the rigidity and stratification of the network \parencite{Kringelbach2023}. However, it is difficult to analyze
these theories directly, and thus far research has failed to
differentiate the action of psychedelics with respect to specific
region- or network-wise changes \parencite{Girn2023}.

Network theory offers a plethora of approaches to the study of the brain
as a complex system, and allows for direct examination of the
hierarchical organization of the brain under psychedelics. Previous
research has pointed toward the idea that the brain, rather than acting
on the world purely through interpretation of environmental stimuli, is
constrained by its own dynamics \parencite{Buzsaki2019}. Irreversibility is then
a measure of how the external environment drives internal brain dynamics \parencite{Buzsaki2019, Deco2022, Kringelbach2023}. By
measuring the irreversibility of brain processes, we can estimate the
extent to which the environment drives intrinsic dynamics, from there
deriving insights about the hierarchical organization of the brain.
Previously, \textcite{Kringelbach2023} showed improved sensitivity of
irreversibility over time-averaged functional connectivity in
differentiating between rest and movie-watching conditions with
effective connectivity derived from irreversibility over functional
connectivity alone, indicating that asymmetry in information flow better
captures features of brain dynamics related to distinct states of
consciousness. A dynamics-based mechanistic
explanation of psychedelic action is especially relevant because of the
preliminary evidence indicating that changes in dynamics predict
psychological and therapeutic outcomes. For example, the degree to which
a participant has a mystical experience in a psychedelic session
predicts both changes in openness to experience 
and increases in entropy \parencite{MacLean2011, Lebedev2016}.

Here we combine the estimation of entropy production in the brain,
irreversibility, with graph theory to examine changes in the functional
hierarchical organization of the brain after ayahuasca in a naturalistic
study with 24 healthy members of Santo Daime, and DMT in a
placebo-controlled study with 17 healthy volunteers with previous
psychedelic experience. In contrast to previous work, users of ayahuasca
in the original study are extreme with regard to lifetime use of
psychedelics (mean = 564, SD = 650). Our main objective was to establish
whether psychedelics had an effect on the functional hierarchical
organization of the brain, and furthermore whether chronic use of
psychedelics alters propensity for changes in hierarchical organization
and baseline structure. In line with the REBUS and entropic brain models
of psychedelic action, we hypothesized that both psychedelics would
result in a flattening of the hierarchy under psychedelics \parencite{Carhart-Harris2014,Carhart-Harris2019}.
Furthermore, we contrast these changes with changes in hierarchical
organization in chronic and occasional users of cannabis. Previous work
in psychedelic research has been limited by the lack of contrast
analyses -- while novel findings are plentiful and leverage a variety of
techniques which converge on similar results, discriminant validity is
lacking. We sought to examine how psychedelics might alter the
hierarchical organization of the brain differently than cannabis in both
chronic and occasional users, especially given the participants under
ayahuasca had considerably more lifetime experience than DMT users.