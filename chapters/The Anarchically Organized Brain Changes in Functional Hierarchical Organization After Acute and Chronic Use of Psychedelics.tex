










\subsection{Limitations (incorporate into
discussion)}\label{limitations-incorporate-into-discussion}

\begin{itemize}
\tightlist
\item
  Different preprocessing methods for DMT vs ayahuasa/cannabis
\item
  Cultural differences
\item
  Variability in dosage within ayahuasca population \& difference in
  dose with DMT
\item
  Structural connectivity obtained from diff population (esp relevant
  for ayahuasca given age diff)
\end{itemize}

\subsection{Appendix}\label{appendix}

TABLE OF ALL THE HIERARCHICAL LEVELS

\begin{longtable}[]{@{}lllll@{}}
\toprule\noalign{}
Resting-state network & Diff & T & p & Sig \\
\midrule\noalign{}
\endhead
\bottomrule\noalign{}
\endlastfoot
\textbf{Ayahuasca} & & & & \\
Visual & 0.021 & 1.04 & 0.1564 & * \\
Somatomotor & 0.023 & 1.49 & 0.0732 & * \\
Dorsal Attention & 0.19 & 4.94 & 0.0001 & * \\
Salience/Ventral Attention & 0.072 & 2.73 & 0.0038 & * \\
Limbic & 0.11 & 4.01 & 0.0003 & * \\
Frontoparietal & 0.10 & 3.083 & 0.0026 & * \\
Default Mode & 0.10 & 3.66 & 0.0004 & * \\
Overall & 0.087 & 4.03 & 6.25e-05 & * \\
\textbf{DMT} & & & & \\
Visual & -0.16 & -3.75 & 0.0001 & * \\
Somatomotor & -0.11 & -3.31 & 0.0003 & * \\
Dorsal Attention & -0.13 & -3.44 & 0.0006 & * \\
Salience/Ventral Attention & -0.09 & -2.96 & 0.0016 & * \\
Limbic & -0.12 & -3.55 & 0.0001 & * \\
Frontoparietal & -0.12 & -3.34 & 0.0009 & * \\
Default Mode & -0.14 & -3.82 & 0.0003 & * \\
Overall & -0.1228 & -3.7199 & 2.27e-04 & * \\
\textbf{Cannabis (Chronic)} & & & & \\
Visual & -0.021 & -1.14 & 0.13 & * \\
Somatomotor & 0.088 & 3.20 & 0.0011 & * \\
Dorsal Attention & 0.021 & 0.89 & 0.19 & * \\
Salience/Ventral Attention & 0.043 & 1.93 & 0.03 & * \\
Limbic & 0.024 & 1.11 & 0.14 & * \\
Frontoparietal & 0.025 & 1.09 & 0.14 & * \\
Default Mode & 0.049 & 2.48 & 0.01 & * \\
Overall & -0.033 & 1.88 & 0.033 & * \\
\textbf{Cannabis (Occasional)} & & & & \\
Visual & 0.074 & 3.50 & 0.0002 & * \\
Somatomotor & 0.065 & 3.16 & 0.0014 & * \\
Dorsal Attention & -0.053 & -2.34 & 0.011 & * \\
Salience/Ventral Attention & -0.020 & -0.88 & 0.20 & \\
Limbic & -0.0002 & 0.19 & 0.43 & \\
Frontoparietal & -0.046 & -2.20 & 0.017 & * \\
Default Mode & -0.0066 & -0.22 & 0.42 & \\
Overall & 0.0099 & 0.89 & 0.19 & \\
\end{longtable}

Table 2: Statistical estimates of the comparison between
baseline/placebo and drug for mean hierarchical levels in resting-state
networks. Positive T value indicates that baseline is greater than
ayahuasca. * indicates that null hypothesis was rejected after false
discovery rate correction with threshold of 0.2. P-vals reported are
minimum after 10,000 iteration, 0.01 threshold non-parametric two-sample
hypothesis test. Overall estimate reported by entering regions that
survived FDR into a second permutation test with 10,000 iterations and
threshold of 0.01.

!{[}{[}Appendix\_ Fits 2.png{]}{]} Supplemental Figure 1: Correlations
between empirical and simulated functional connectivity (FC) and
covariance matrices. Correlations were taken for each participant after
model convergence with the MATLAB \texttt{corr2} function. ``1'' denotes
baseline or placebo conditions. ``2'' denotes drug conditions.

!{[}{[}hlayafull.png{]}{]} Supplemental Figure 2: Hierarchical level
changes for each region under baseline and ayahuasca.

Supplemental Figure x:

\subsection*{References}\label{references}
\addcontentsline{toc}{subsection}{References}

\phantomsection\label{refs}
\begin{CSLReferences}{1}{0}
\bibitem[\citeproctext]{ref-Andersson2001}
Andersson, J. L., Hutton, C., Ashburner, J., Turner, R., \& Friston, K.
(2001). Modeling geometric deformations in EPI time series.
\emph{NeuroImage}, \emph{13}(5), 903--919.
\url{https://doi.org/10.1006/nimg.2001.0746}

\bibitem[\citeproctext]{ref-Ashburner2007}
Ashburner, J. (2007). A fast diffeomorphic image registration algorithm.
\emph{NeuroImage}, \emph{38}(1), 95--113.
\url{https://doi.org/10.1016/j.neuroimage.2007.07.007}

\bibitem[\citeproctext]{ref-Ashburner2005}
Ashburner, J., \& Friston, K. J. (2005). Unified segmentation.
\emph{NeuroImage}, \emph{26}(3), 839--851.
\url{https://doi.org/10.1016/j.neuroimage.2005.02.018}

\bibitem[\citeproctext]{ref-Atasoy2017}
Atasoy, S., Roseman, L., Kaelen, M., Kringelbach, M. L., Deco, G., \&
Carhart-Harris, R. L. (2017). Connectome-harmonic decomposition of human
brain activity reveals dynamical repertoire re-organization under LSD.
\emph{Scientific Reports}, \emph{7}(1), Article 1.
\url{https://doi.org/10.1038/s41598-017-17546-0}

\bibitem[\citeproctext]{ref-Avants2009}
Avants, B. B., Tustison, N. J., \& Song, G. (2009). Advanced
Normalization Tools: V1.0. \emph{The Insight Journal}.
\url{https://doi.org/10.54294/uvnhin}

\bibitem[\citeproctext]{ref-Barbosa2009}
Barbosa, P. C. R., Cazorla, I. M., Giglio, J. S., \& Strassman, R.
(2009). A six-month prospective evaluation of personality traits,
psychiatric symptoms and quality of life in ayahuasca-naïve subjects.
\emph{Journal of Psychoactive Drugs}, \emph{41}(3), 205--212.
\url{https://doi.org/10.1080/02791072.2009.10400530}

\bibitem[\citeproctext]{ref-Barrett2020}
Barrett, F. S., Doss, M. K., Sepeda, N. D., Pekar, J. J., \& Griffiths,
R. R. (2020). Emotions and brain function are altered up to one month
after a single high dose of psilocybin. \emph{Scientific Reports},
\emph{10}(1), Article 1.
\url{https://doi.org/10.1038/s41598-020-59282-y}

\bibitem[\citeproctext]{ref-Beggs2003}
Beggs, J. M., \& Plenz, D. (2003). Neuronal avalanches in neocortical
circuits. \emph{The Journal of Neuroscience: The Official Journal of the
Society for Neuroscience}, \emph{23}(35), 11167--11177.
\url{https://doi.org/10.1523/JNEUROSCI.23-35-11167.2003}

\bibitem[\citeproctext]{ref-Behzadi2007}
Behzadi, Y., Restom, K., Liau, J., \& Liu, T. T. (2007). A component
based noise correction method (CompCor) for BOLD and perfusion based
fMRI. \emph{NeuroImage}, \emph{37}(1), 90--101.
\url{https://doi.org/10.1016/j.neuroimage.2007.04.042}

\bibitem[\citeproctext]{ref-Benjamini1995}
Benjamini, Y., \& Hochberg, Y. (1995). Controlling the False Discovery
Rate: A Practical and Powerful Approach to Multiple Testing.
\emph{Journal of the Royal Statistical Society: Series B
(Methodological)}, \emph{57}(1), 289--300.
\url{https://doi.org/10.1111/j.2517-6161.1995.tb02031.x}

\bibitem[\citeproctext]{ref-Bogenschutz2015}
Bogenschutz, M. P., Forcehimes, A. A., Pommy, J. A., Wilcox, C. E.,
Barbosa, P. C. R., \& Strassman, R. J. (2015). Psilocybin-assisted
treatment for alcohol dependence: a proof-of-concept study.
\emph{Journal of Psychopharmacology (Oxford, England)}, \emph{29}(3),
289--299. \url{https://doi.org/10.1177/0269881114565144}

\bibitem[\citeproctext]{ref-Bouso2015}
Bouso, J. C., Palhano-Fontes, F., Rodríguez-Fornells, A., Ribeiro, S.,
Sanches, R. F., de Souza Crippa, J. A., Hallak, J. E. C., de Araújo, D.
B., \& Riba, J. (2015). Long-term use of psychedelic drugs is associated
with differences in brain structure and personality in humans \$.
\emph{European Neuropsychopharmacology}, \emph{25}(4), 483--492.
\url{https://doi.org/10.1016/j.euroneuro.2015.01.008}

\bibitem[\citeproctext]{ref-Branchi2011}
Branchi, I. (2011). The double edged sword of neural plasticity:
increasing serotonin levels leads to both greater vulnerability to
depression and improved capacity to recover.
\emph{Psychoneuroendocrinology}, \emph{36}(3), 339--351.
\url{https://doi.org/10.1016/j.psyneuen.2010.08.011}

\bibitem[\citeproctext]{ref-Buckner2008}
Buckner, R. L., Andrews-Hanna, J. R., \& Schacter, D. L. (2008).
\emph{The Brain's Default Network}: \emph{Anatomy, Function, and
Relevance to Disease}. \emph{Annals of the New York Academy of
Sciences}, \emph{1124}(1), 1--38.
\url{https://doi.org/10.1196/annals.1440.011}

\bibitem[\citeproctext]{ref-Buzsaki2019}
Buzsáki, G. (2019). \emph{The Brain From Inside Out}.
\url{https://academic.oup.com/book/35081}

\bibitem[\citeproctext]{ref-Calhoun2017}
Calhoun, V. D., Wager, T. D., Krishnan, A., Rosch, K. S., Seymour, K.
E., Nebel, M. B., Mostofsky, S. H., Nyalakanai, P., \& Kiehl, K. (2017).
The impact of T1 versus EPI spatial normalization templates for fMRI
data analyses. \emph{Human Brain Mapping}, \emph{38}(11), 5331--5342.
\url{https://doi.org/10.1002/hbm.23737}

\bibitem[\citeproctext]{ref-Capouskova2022}
Capouskova, K., Kringelbach, M. L., \& Deco, G. (2022). Modes of
cognition: Evidence from metastable brain dynamics. \emph{NeuroImage},
\emph{260}, 119489.
\url{https://doi.org/10.1016/j.neuroimage.2022.119489}

\bibitem[\citeproctext]{ref-Carhart-Harris2007}
Carhart-Harris, R. (2007). Waves of the Unconscious: The Neurophysiology
of Dreamlike Phenomena and Its Implications for the Psychodynamic Model
of the Mind. \emph{Neuropsychoanalysis}, \emph{9}(2), 183--211.
\url{https://doi.org/10.1080/15294145.2007.10773557}

\bibitem[\citeproctext]{ref-Carhart-Harris2018}
Carhart-Harris, R. L. (2018). The entropic brain - revisited.
\emph{Neuropharmacology}, \emph{142}, 167--178.
\url{https://doi.org/10.1016/j.neuropharm.2018.03.010}

\bibitem[\citeproctext]{ref-Carhart-Harris2019}
Carhart-Harris, R. L. (2019). How do psychedelics work? \emph{Current
Opinion in Psychiatry}, \emph{32}(1), 16--21.
\url{https://doi.org/10.1097/YCO.0000000000000467}

\bibitem[\citeproctext]{ref-Carhart-Harris2012}
Carhart-Harris, R. L., Erritzoe, D., Williams, T., Stone, J. M., Reed,
L. J., Colasanti, A., Tyacke, R. J., Leech, R., Malizia, A. L., Murphy,
K., Hobden, P., Evans, J., Feilding, A., Wise, R. G., \& Nutt, D. J.
(2012). Neural correlates of the psychedelic state as determined by fMRI
studies with psilocybin. \emph{Proceedings of the National Academy of
Sciences of the United States of America}, \emph{109}(6), 2138--2143.
\url{https://doi.org/10.1073/pnas.1119598109}

\bibitem[\citeproctext]{ref-Carhart-Harris2019a}
Carhart-Harris, R. L., \& Friston, K. J. (2019). REBUS and the Anarchic
Brain: Toward a Unified Model of the Brain Action of Psychedelics.
\emph{Pharmacological Reviews}, \emph{71}(3), 316--344.
\url{https://doi.org/10.1124/pr.118.017160}

\bibitem[\citeproctext]{ref-Carhart-Harris2017}
Carhart-Harris, R. L., Roseman, L., Bolstridge, M., Demetriou, L.,
Pannekoek, J. N., Wall, M. B., Tanner, M. A., Kaelen, M., McGonigle, J.,
Murphy, K., Leech, R., Curran, H. V., \& Nutt, D. J. (2017). Psilocybin
for treatment-resistant depression: fMRI-measured brain mechanisms.
\emph{Scientific Reports}, \emph{7}(1).
\url{https://doi.org/10.1038/s41598-017-13282-7}

\bibitem[\citeproctext]{ref-Carhart-Harris2021}
Carhart-Harris, R., Giribaldi, B., Watts, R., Baker-Jones, M.,
Murphy-Beiner, A., Murphy, R., Martell, J., Blemings, A., Erritzoe, D.,
\& Nutt, D. J. (2021). Trial of Psilocybin versus Escitalopram for
Depression. \emph{The New England Journal of Medicine}, \emph{384}(15),
1402--1411. \url{https://doi.org/10.1056/NEJMoa2032994}

\bibitem[\citeproctext]{ref-Carhart-Harris2014}
Carhart-Harris, R., Leech, R., Hellyer, P., Shanahan, M., Feilding, A.,
Tagliazucchi, E., Chialvo, D., \& Nutt, D. (2014). The entropic brain: a
theory of conscious states informed by neuroimaging research with
psychedelic drugs. \emph{Frontiers in Human Neuroscience}, \emph{8}.
\url{https://www.frontiersin.org/articles/10.3389/fnhum.2014.00020}

\bibitem[\citeproctext]{ref-Carhart-Harris2017a}
Carhart-Harris, R., \& Nutt, D. (2017). Serotonin and brain function: a
tale of two receptors. \emph{Journal of Psychopharmacology},
\emph{31}(9), 1091--1120. \url{https://doi.org/10.1177/0269881117725915}

\bibitem[\citeproctext]{ref-Chai2012}
Chai, X. J., Castañón, A. N., Ongür, D., \& Whitfield-Gabrieli, S.
(2012). Anticorrelations in resting state networks without global signal
regression. \emph{NeuroImage}, \emph{59}(2), 1420--1428.
\url{https://doi.org/10.1016/j.neuroimage.2011.08.048}

\bibitem[\citeproctext]{ref-Cox1996}
Cox, R. W. (1996). AFNI: software for analysis and visualization of
functional magnetic resonance neuroimages. \emph{Computers and
Biomedical Research, an International Journal}, \emph{29}(3), 162--173.
\url{https://doi.org/10.1006/cbmr.1996.0014}

\bibitem[\citeproctext]{ref-Dale1999}
Dale, A. M., Fischl, B., \& Sereno, M. I. (1999). Cortical Surface-Based
Analysis. \emph{NeuroImage}, \emph{9}(2).
\url{https://doi.org/10.1006/nimg.1998.0395}

\bibitem[\citeproctext]{ref-Daws2022}
Daws, R. E., Timmermann, C., Giribaldi, B., Sexton, J. D., Wall, M. B.,
Erritzoe, D., Roseman, L., Nutt, D., \& Carhart-Harris, R. (2022).
Increased global integration in the brain after psilocybin therapy for
depression. \emph{Nature Medicine}, \emph{28}(4), Article 4.
\url{https://doi.org/10.1038/s41591-022-01744-z}

\bibitem[\citeproctext]{ref-Deco2019a}
Deco, G., Cruzat, J., Cabral, J., Tagliazucchi, E., Laufs, H.,
Logothetis, N. K., \& Kringelbach, M. L. (2019). Awakening: Predicting
external stimulation to force transitions between different brain
states. \emph{Proceedings of the National Academy of Sciences},
\emph{116}(36), 18088--18097.
\url{https://doi.org/10.1073/pnas.1905534116}

\bibitem[\citeproctext]{ref-Deco2019}
Deco, G., Cruzat, J., \& Kringelbach, M. L. (2019). Brain songs
framework used for discovering the relevant timescale of the human
brain. \emph{Nature Communications}, \emph{10}(1), Article 1.
\url{https://doi.org/10.1038/s41467-018-08186-7}

\bibitem[\citeproctext]{ref-Deco2017b}
Deco, G., \& Kringelbach, M. L. (2017). Hierarchy of Information
Processing in the Brain: A Novel {``Intrinsic Ignition''} Framework.
\emph{Neuron}, \emph{94}(5), 961--968.
\url{https://doi.org/10.1016/j.neuron.2017.03.028}

\bibitem[\citeproctext]{ref-Deco2017}
Deco, G., Kringelbach, M. L., Jirsa, V. K., \& Ritter, P. (2017). The
dynamics of resting fluctuations in the brain: metastability and its
dynamical cortical core. \emph{Scientific Reports}, \emph{7}(1), Article
1. \url{https://doi.org/10.1038/s41598-017-03073-5}

\bibitem[\citeproctext]{ref-Deco2021a}
Deco, G., Perl, Y. S., Sitt, J. D., Tagliazucchi, E., \& Kringelbach, M.
L. (2021). \emph{Deep learning the arrow of time in brain activity:
characterising brain-environment behavioural interactions in health and
disease} (p. 2021.07.02.450899). bioRxiv.
\url{https://doi.org/10.1101/2021.07.02.450899}

\bibitem[\citeproctext]{ref-Deco2022}
Deco, G., Sanz Perl, Y., Bocaccio, H., Tagliazucchi, E., \& Kringelbach,
M. L. (2022). The INSIDEOUT framework provides precise signatures of the
balance of intrinsic and extrinsic dynamics in brain states.
\emph{Communications Biology}, \emph{5}(1), Article 1.
\url{https://doi.org/10.1038/s42003-022-03505-7}

\bibitem[\citeproctext]{ref-Deco2021}
Deco, G., Vidaurre, D., \& Kringelbach, M. L. (2021). Revisiting the
global workspace orchestrating the hierarchical organization of the
human brain. \emph{Nature Human Behaviour}, \emph{5}(4), Article 4.
\url{https://doi.org/10.1038/s41562-020-01003-6}

\bibitem[\citeproctext]{ref-Desikan2006}
Desikan, R. S., Ségonne, F., Fischl, B., Quinn, B. T., Dickerson, B. C.,
Blacker, D., Buckner, R. L., Dale, A. M., Maguire, R. P., Hyman, B. T.,
Albert, M. S., \& Killiany, R. J. (2006). An automated labeling system
for subdividing the human cerebral cortex on MRI scans into gyral based
regions of interest. \emph{NeuroImage}, \emph{31}(3), 968--980.
\url{https://doi.org/10.1016/j.neuroimage.2006.01.021}

\bibitem[\citeproctext]{ref-Friston2003}
Friston, K. J., Harrison, L., \& Penny, W. (2003). Dynamic causal
modelling. \emph{NeuroImage}, \emph{19}(4), 1273--1302.
\url{https://doi.org/10.1016/S1053-8119(03)00202-7}

\bibitem[\citeproctext]{ref-Friston1996}
Friston, K. J., Williams, S., Howard, R., Frackowiak, R. S., \& Turner,
R. (1996). Movement-related effects in fMRI time-series. \emph{Magnetic
Resonance in Medicine}, \emph{35}(3), 346--355.
\url{https://doi.org/10.1002/mrm.1910350312}

\bibitem[\citeproctext]{ref-Friston1995}
Friston, Karl. J., Ashburner, J., Frith, C. D., Poline, J.-B., Heather,
J. D., \& Frackowiak, R. S. J. (1995). Spatial registration and
normalization of images. \emph{Human Brain Mapping}, \emph{3}(3),
165--189. \url{https://doi.org/10.1002/hbm.460030303}

\bibitem[\citeproctext]{ref-Girn2023}
Girn, M., Rosas, F. E., Daws, R. E., Gallen, C. L., Gazzaley, A., \&
Carhart-Harris, R. L. (2023). A complex systems perspective on
psychedelic brain action. \emph{Trends in Cognitive Sciences},
\emph{27}(5), 433--445. \url{https://doi.org/10.1016/j.tics.2023.01.003}

\bibitem[\citeproctext]{ref-Girn2022}
Girn, M., Roseman, L., Bernhardt, B., Smallwood, J., Carhart-Harris, R.,
\& Nathan Spreng, R. (2022). Serotonergic psychedelic drugs LSD and
psilocybin reduce the hierarchical differentiation of unimodal and
transmodal cortex. \emph{NeuroImage}, \emph{256}, 119220.
\url{https://doi.org/10.1016/j.neuroimage.2022.119220}

\bibitem[\citeproctext]{ref-Glasser2016}
Glasser, M. F., Coalson, T. S., Robinson, E. C., Hacker, C. D., Harwell,
J. W., Yacoub, E., Ugurbil, K., Andersson, J. L. R., Beckmann, C. F.,
Jenkinson, M., Smith, S., \& Van Essen, D. C. (2016). A multi-modal
parcellation of human cerebral cortex. \emph{Nature}, \emph{536}(7615),
171--178. \url{https://doi.org/10.1038/nature18933}

\bibitem[\citeproctext]{ref-Golesorkhi2021}
Golesorkhi, M., Gomez-Pilar, J., Zilio, F., Berberian, N., Wolff, A.,
Yagoub, M. C. E., \& Northoff, G. (2021). The brain and its time:
intrinsic neural timescales are key for input processing.
\emph{Communications Biology}, \emph{4}(1), Article 1.
\url{https://doi.org/10.1038/s42003-021-02483-6}

\bibitem[\citeproctext]{ref-Gomes2020}
Gomes, M. T., Fernandes, H. M., \& Cabral, J. (2020). \emph{Deep brain
stimulation modulates the dynamics of resting-state networks in patients
with Parkinson's Disease} (p. 2020.11.04.368274). bioRxiv.
\url{https://doi.org/10.1101/2020.11.04.368274}

\bibitem[\citeproctext]{ref-Griffiths2011}
Griffiths, R. R., Johnson, M. W., Richards, W. A., Richards, B. D.,
McCann, U., \& Jesse, R. (2011). Psilocybin occasioned mystical-type
experiences: immediate and persisting dose-related effects.
\emph{Psychopharmacology}, \emph{218}(4), 649--665.
\url{https://doi.org/10.1007/s00213-011-2358-5}

\bibitem[\citeproctext]{ref-Griffiths2006}
Griffiths, R. R., Richards, W. A., McCann, U., \& Jesse, R. (2006).
Psilocybin can occasion mystical-type experiences having substantial and
sustained personal meaning and spiritual significance.
\emph{Psychopharmacology}, \emph{187}(3), 268--283.
\url{https://doi.org/10.1007/s00213-006-0457-5}

\bibitem[\citeproctext]{ref-Griffiths2008}
Griffiths, R., Richards, W., Johnson, M., McCann, U., \& Jesse, R.
(2008). Mystical-type experiences occasioned by psilocybin mediate the
attribution of personal meaning and spiritual significance 14 months
later. \emph{Journal of Psychopharmacology}, \emph{22}(6), 621--632.
\url{https://doi.org/10.1177/0269881108094300}

\bibitem[\citeproctext]{ref-Grob2011}
Grob, C. S., Danforth, A. L., Chopra, G. S., Hagerty, M., McKay, C. R.,
Halberstadt, A. L., \& Greer, G. R. (2011). Pilot study of psilocybin
treatment for anxiety in patients with advanced-stage cancer.
\emph{Archives of General Psychiatry}, \emph{68}(1), 71--78.
\url{https://doi.org/10.1001/archgenpsychiatry.2010.116}

\bibitem[\citeproctext]{ref-Hallquist2013}
Hallquist, M. N., Hwang, K., \& Luna, B. (2013). The Nuisance of
Nuisance Regression: Spectral Misspecification in a Common Approach to
Resting-State fMRI Preprocessing Reintroduces Noise and Obscures
Functional Connectivity. \emph{NeuroImage}, \emph{0}, 208--225.
\url{https://doi.org/10.1016/j.neuroimage.2013.05.116}

\bibitem[\citeproctext]{ref-Henson1999}
Henson, R., Buechel, C., Josephs, O., \& Friston, K. J. (1999). The
slice-timing problem in event-related fMRI. \emph{NeuroImage}.
\url{https://www.semanticscholar.org/paper/The-slice-timing-problem-in-event-related-fMRI-Henson-Buechel/e257acebf8bbda0811d452b277208797c4b12506}

\bibitem[\citeproctext]{ref-Johnson2014a}
Johnson, M. W., Garcia-Romeu, A., Cosimano, M. P., \& Griffiths, R. R.
(2014). Pilot study of the 5-HT2AR agonist psilocybin in the treatment
of tobacco addiction. \emph{Journal of Psychopharmacology},
\emph{28}(11), Article 11.
\url{https://doi.org/10.1177/0269881114548296}

\bibitem[\citeproctext]{ref-Johnson2017}
Johnson, M. W., \& Griffiths, R. R. (2017). Potential Therapeutic
Effects of Psilocybin. \emph{Neurotherapeutics: The Journal of the
American Society for Experimental NeuroTherapeutics}, \emph{14}(3),
734--740. \url{https://doi.org/10.1007/s13311-017-0542-y}

\bibitem[\citeproctext]{ref-Johnson2019}
Johnson, M. W., Hendricks, P. S., Barrett, F. S., \& Griffiths, R. R.
(2019). Classic psychedelics: An integrative review of epidemiology,
therapeutics, mystical experience, and brain network function.
\emph{Pharmacology \& Therapeutics}, \emph{197}, 83--102.
\url{https://doi.org/10.1016/j.pharmthera.2018.11.010}

\bibitem[\citeproctext]{ref-Johnson2014}
Johnson, S., Domínguez-García, V., Donetti, L., \& Muñoz, M. A. (2014).
Trophic coherence determines food-web stability. \emph{Proceedings of
the National Academy of Sciences of the United States of America},
\emph{111}(50), 17923--17928.
\url{https://doi.org/10.1073/pnas.1409077111}

\bibitem[\citeproctext]{ref-Klein2012}
Klein, A., \& Tourville, J. (2012). 101 Labeled Brain Images and a
Consistent Human Cortical Labeling Protocol. \emph{Frontiers in
Neuroscience}, \emph{6}.
\url{https://www.frontiersin.org/articles/10.3389/fnins.2012.00171}

\bibitem[\citeproctext]{ref-Kobeleva2021}
Kobeleva, X., López-González, A., Kringelbach, M. L., \& Deco, G.
(2021). Revealing the Relevant Spatiotemporal Scale Underlying
Whole-Brain Dynamics. \emph{Frontiers in Neuroscience}, \emph{15}.
\url{https://www.frontiersin.org/articles/10.3389/fnins.2021.715861}

\bibitem[\citeproctext]{ref-Kometer2016}
Kometer, M., \& Vollenweider, F. X. (2016). \emph{Serotonergic
Hallucinogen-Induced Visual Perceptual Alterations}. \emph{36},
257--282. \url{https://doi.org/10.1007/7854_2016_461}

\bibitem[\citeproctext]{ref-Kraehenmann2017}
Kraehenmann, R. (2017). Dreams and Psychedelics: Neurophenomenological
Comparison and Therapeutic Implications. \emph{Current
Neuropharmacology}, \emph{15}(7), 1032--1042.
\url{https://doi.org/10.2174/1573413713666170619092629}

\bibitem[\citeproctext]{ref-Kraehenmann2017a}
Kraehenmann, R., Pokorny, D., Aicher, H., Preller, K. H., Pokorny, T.,
Bosch, O. G., Seifritz, E., \& Vollenweider, F. X. (2017). LSD Increases
Primary Process Thinking via Serotonin 2A Receptor Activation.
\emph{Frontiers in Pharmacology}, \emph{8}, 814.
\url{https://doi.org/10.3389/fphar.2017.00814}

\bibitem[\citeproctext]{ref-Kringelbach2020}
Kringelbach, M. L., \& Deco, G. (2020). Brain States and Transitions:
Insights from Computational Neuroscience. \emph{Cell Reports},
\emph{32}(10), 108128.
\url{https://doi.org/10.1016/j.celrep.2020.108128}

\bibitem[\citeproctext]{ref-Kringelbach2023}
Kringelbach, M. L., Perl, Y. S., Tagliazucchi, E., \& Deco, G. (2023).
Toward naturalistic neuroscience: Mechanisms underlying the flattening
of brain hierarchy in movie-watching compared to rest and task.
\emph{Science Advances}, \emph{9}(2), eade6049.
\url{https://doi.org/10.1126/sciadv.ade6049}

\bibitem[\citeproctext]{ref-Lebedev2016}
Lebedev, A. v., Kaelen, M., Lövdén, M., Nilsson, J., Feilding, A., Nutt,
D. j., \& Carhart-Harris, R. l. (2016). LSD-induced entropic brain
activity predicts subsequent personality change. \emph{Human Brain
Mapping}, \emph{37}(9), 3203--3213.
\url{https://doi.org/10.1002/hbm.23234}

\bibitem[\citeproctext]{ref-Lebedev2015}
Lebedev, A. V., Lövdén, M., Rosenthal, G., Feilding, A., Nutt, D. J., \&
Carhart-Harris, R. L. (2015). Finding the self by losing the self:
Neural correlates of ego-dissolution under psilocybin. \emph{Human Brain
Mapping}, \emph{36}(8), 3137--3153.
\url{https://doi.org/10.1002/hbm.22833}

\bibitem[\citeproctext]{ref-Lord2019}
Lord, L.-D., Expert, P., Atasoy, S., Roseman, L., Rapuano, K.,
Lambiotte, R., Nutt, D. J., Deco, G., Carhart-Harris, R. L.,
Kringelbach, M. L., \& Cabral, J. (2019). Dynamical exploration of the
repertoire of brain networks at rest is modulated by psilocybin.
\emph{NeuroImage}, \emph{199}, 127--142.
\url{https://doi.org/10.1016/j.neuroimage.2019.05.060}

\bibitem[\citeproctext]{ref-Luppi2021a}
Luppi, A. I., Mediano, P. A. M., Rosas, F., Harrison, D. J.,
Carhart-Harris, R. L., Bor, D., \& Stamatakis, E. A. (2021). \emph{What
it is like to be a bit: An Integrated Information Decomposition account
of emergent mental phenomena}. PsyArXiv.
\url{https://doi.org/10.31234/osf.io/g9p3r}

\bibitem[\citeproctext]{ref-Lynn2021}
Lynn, C. W., Cornblath, E. J., Papadopoulos, L., Bertolero, M. A., \&
Bassett, D. S. (2021). Broken detailed balance and entropy production in
the human brain. \emph{Proceedings of the National Academy of Sciences
of the United States of America}, \emph{118}(47), e2109889118.
\url{https://doi.org/10.1073/pnas.2109889118}

\bibitem[\citeproctext]{ref-MacKay2020}
MacKay, R. S., Johnson, S., \& Sansom, B. (2020). How directed is a
directed network? \emph{Royal Society Open Science}, \emph{7}(9),
201138. \url{https://doi.org/10.1098/rsos.201138}

\bibitem[\citeproctext]{ref-MacLean2011}
MacLean, K. A., Johnson, M. W., \& Griffiths, R. R. (2011). Mystical
Experiences Occasioned by the Hallucinogen Psilocybin Lead to Increases
in the Personality Domain of Openness. \emph{Journal of
Psychopharmacology (Oxford, England)}, \emph{25}(11), 1453--1461.
\url{https://doi.org/10.1177/0269881111420188}

\bibitem[\citeproctext]{ref-Mallaroni2022}
Mallaroni, P., Mason, N. L., Kloft, L., Reckweg, J. T., Oorsouw, K. van,
Toennes, S. W., Tolle, H. M., Amico, E., \& Ramaekers, J. G. (2022).
\emph{Ritualistic use of ayahuasca enhances a shared functional
connectome identity with others} (p. 2022.10.07.511268). bioRxiv.
\url{https://doi.org/10.1101/2022.10.07.511268}

\bibitem[\citeproctext]{ref-McCrae1997}
McCrae, R. R., \& Costa Jr., P. T. (1997). Personality trait structure
as a human universal. \emph{American Psychologist}, \emph{52}(5),
509--516. \url{https://doi.org/10.1037/0003-066X.52.5.509}

\bibitem[\citeproctext]{ref-McCulloch2022}
McCulloch, D. E.-W., Knudsen, G. M., Barrett, F. S., Doss, M. K.,
Carhart-Harris, R. L., Rosas, F. E., Deco, G., Kringelbach, M. L.,
Preller, K. H., Ramaekers, J. G., Mason, N. L., Müller, F., \& Fisher,
P. M. (2022). Psychedelic resting-state neuroimaging: A review and
perspective on balancing replication and novel analyses.
\emph{Neuroscience \& Biobehavioral Reviews}, \emph{138}, 104689.
\url{https://doi.org/10.1016/j.neubiorev.2022.104689}

\bibitem[\citeproctext]{ref-Moliner2023}
Moliner, R., Girych, M., Brunello, C. A., Kovaleva, V., Biojone, C.,
Enkavi, G., Antenucci, L., Kot, E. F., Goncharuk, S. A., Kaurinkoski,
K., Kuutti, M., Fred, S. M., Elsilä, L. V., Sakson, S., Cannarozzo, C.,
Diniz, C. R. A. F., Seiffert, N., Rubiolo, A., Haapaniemi, H., \ldots{}
Castrén, E. (2023). Psychedelics promote plasticity by directly binding
to BDNF receptor TrkB. \emph{Nature Neuroscience}, \emph{26}(6), Article
6. \url{https://doi.org/10.1038/s41593-023-01316-5}

\bibitem[\citeproctext]{ref-Moreno2006}
Moreno, F. A., Wiegand, C. B., Taitano, E. K., \& Delgado, P. L. (2006).
Safety, tolerability, and efficacy of psilocybin in 9 patients with
obsessive-compulsive disorder. \emph{The Journal of Clinical
Psychiatry}, \emph{67}(11), 1735--1740.
\url{https://doi.org/10.4088/jcp.v67n1110}

\bibitem[\citeproctext]{ref-Mueller2013}
Mueller, S., Wang, D., Fox, M. D., Thomas Yeo, B. T., Sepulcre, J.,
Sabuncu, M. R., Shafee, R., Lu, J., \& Liu, H. (2013). Individual
Variability in Functional Connectivity Architecture of the Human Brain.
\emph{Neuron}, \emph{77}(3), 586--595.
\url{https://doi.org/10.1016/j.neuron.2012.12.028}

\bibitem[\citeproctext]{ref-Muller2018}
Müller, F., Dolder, P. C., Schmidt, A., Liechti, M. E., \& Borgwardt, S.
(2018). Altered network hub connectivity after acute LSD administration.
\emph{NeuroImage: Clinical}, \emph{18}, 694--701.
\url{https://doi.org/10.1016/j.nicl.2018.03.005}

\bibitem[\citeproctext]{ref-Nichols2016}
Nichols, D. E. (2016). Psychedelics. \emph{Pharmacological Reviews},
\emph{68}(2), 264--355. \url{https://doi.org/10.1124/pr.115.011478}

\bibitem[\citeproctext]{ref-Nieto-Castanon2020}
Nieto-Castanon, A. (2020). \emph{Handbook of functional connectivity
Magnetic Resonance Imaging methods in CONN}.
\url{https://doi.org/10.56441/hilbertpress.2207.6598}

\bibitem[\citeproctext]{ref-Nieto-Castanon2022}
Nieto-Castanon, A. (2022). \emph{Preparing fMRI Data for Statistical
Analysis} (arXiv:2210.13564). arXiv.
\url{https://doi.org/10.48550/arXiv.2210.13564}

\bibitem[\citeproctext]{ref-Nieto-Castanon2022a}
Nieto-Castanon, A., \& Whitfield-Gabrieli, S. (2022). \emph{CONN
functional connectivity toolbox: RRID SCR\_009550, release 22} (RRID
SCR\_009550, release 22) {[}Computer software{]}.
\url{https://www.hilbertpress.org/link-nieto-castanon2022}

\bibitem[\citeproctext]{ref-Penny2007}
Penny, W., Friston, K. J., Ashburner, J., Kiebel, S., \& Nichols, T. E.
(2007). \emph{Statistical Parametric Mapping: The Analysis of Functional
Brain Images}.
\url{https://www.semanticscholar.org/paper/Statistical-Parametric-Mapping\%3A-The-Analysis-of-Penny-Friston/559f06ecfe15b6f994ab6f685e9293cd43947550}

\bibitem[\citeproctext]{ref-Petri2014}
Petri, G., Expert, P., Turkheimer, F., Carhart-Harris, R., Nutt, D.,
Hellyer, P. J., \& Vaccarino, F. (2014). Homological scaffolds of brain
functional networks. \emph{Journal of The Royal Society Interface},
\emph{11}(101), 20140873. \url{https://doi.org/10.1098/rsif.2014.0873}

\bibitem[\citeproctext]{ref-Power2014}
Power, J. D., Mitra, A., Laumann, T. O., Snyder, A. Z., Schlaggar, B.
L., \& Petersen, S. E. (2014). Methods to detect, characterize, and
remove motion artifact in resting state fMRI. \emph{NeuroImage},
\emph{84}, 320--341.
\url{https://doi.org/10.1016/j.neuroimage.2013.08.048}

\bibitem[\citeproctext]{ref-Raison2023}
Raison, C. L., Sanacora, G., Woolley, J., Heinzerling, K., Dunlop, B.
W., Brown, R. T., Kakar, R., Hassman, M., Trivedi, R. P., Robison, R.,
Gukasyan, N., Nayak, S. M., Hu, X., O'Donnell, K. C., Kelmendi, B.,
Sloshower, J., Penn, A. D., Bradley, E., Kelly, D. F., \ldots{}
Griffiths, R. R. (2023). Single-Dose Psilocybin Treatment for Major
Depressive Disorder: A Randomized Clinical Trial. \emph{JAMA},
\emph{330}(9), 843--853. \url{https://doi.org/10.1001/jama.2023.14530}

\bibitem[\citeproctext]{ref-Ramaekers2020}
Ramaekers, J. G., Mason, N. L., \& Theunissen, E. L. (2020). Blunted
highs: Pharmacodynamic and behavioral models of cannabis tolerance.
\emph{European Neuropsychopharmacology: The Journal of the European
College of Neuropsychopharmacology}, \emph{36}, 191--205.
\url{https://doi.org/10.1016/j.euroneuro.2020.01.006}

\bibitem[\citeproctext]{ref-Ramaekers2022}
Ramaekers, J., Mason, N., Toennes, S., Theunissen, E., \& Amico, E.
(2022). Functional brain connectomes reflect acute and chronic cannabis
use. \emph{Scientific Reports}, \emph{12}.
\url{https://doi.org/10.1038/s41598-022-06509-9}

\bibitem[\citeproctext]{ref-Riba2006}
Riba, J., Romero, S., Grasa, E., Mena, E., Carrió, I., \& Barbanoj, M.
J. (2006). Increased frontal and paralimbic activation following
ayahuasca, the pan-Amazonian inebriant. \emph{Psychopharmacology},
\emph{186}(1), 93--98. \url{https://doi.org/10.1007/s00213-006-0358-7}

\bibitem[\citeproctext]{ref-Roseman2019}
Roseman, L., Haijen, E., Idialu-Ikato, K., Kaelen, M., Watts, R., \&
Carhart-Harris, R. (2019). Emotional breakthrough and psychedelics:
Validation of the Emotional Breakthrough Inventory. \emph{Journal of
Psychopharmacology (Oxford, England)}, \emph{33}(9), 1076--1087.
\url{https://doi.org/10.1177/0269881119855974}

\bibitem[\citeproctext]{ref-Roseman2014}
Roseman, L., Leech, R., Feilding, A., Nutt, D. J., \& Carhart-Harris, R.
L. (2014). The effects of psilocybin and MDMA on between-network resting
state functional connectivity in healthy volunteers. \emph{Frontiers in
Human Neuroscience}, \emph{8}.
\url{https://doi.org/10.3389/fnhum.2014.00204}

\bibitem[\citeproctext]{ref-SanzPerl2021}
Sanz Perl, Y., Bocaccio, H., Pallavicini, C., Pérez-Ipiña, I., Laureys,
S., Laufs, H., Kringelbach, M., Deco, G., \& Tagliazucchi, E. (2021).
Nonequilibrium brain dynamics as a signature of consciousness.
\emph{Physical Review. E}, \emph{104}(1-1), 014411.
\url{https://doi.org/10.1103/PhysRevE.104.014411}

\bibitem[\citeproctext]{ref-Seif2021}
Seif, A., Hafezi, M., \& Jarzynski, C. (2021). Machine learning the
thermodynamic arrow of time. \emph{Nature Physics}, \emph{17}(1),
Article 1. \url{https://doi.org/10.1038/s41567-020-1018-2}

\bibitem[\citeproctext]{ref-Setsompop2013}
Setsompop, K., Kimmlingen, R., Eberlein, E., Witzel, T., Cohen-Adad, J.,
McNab, J. A., Keil, B., Tisdall, M. D., Hoecht, P., Dietz, P., Cauley,
S. F., Tountcheva, V., Matschl, V., Lenz, V. H., Heberlein, K.,
Potthast, A., Thein, H., Van Horn, J., Toga, A., \ldots{} Wald, L. L.
(2013). Pushing the limits of in vivo diffusion MRI for the Human
Connectome Project. \emph{NeuroImage}, \emph{80}, 220--233.
\url{https://doi.org/10.1016/j.neuroimage.2013.05.078}

\bibitem[\citeproctext]{ref-Singleton2022}
Singleton, S. P., Luppi, A. I., Carhart-Harris, R. L., Cruzat, J.,
Roseman, L., Nutt, D. J., Deco, G., Kringelbach, M. L., Stamatakis, E.
A., \& Kuceyeski, A. (2022). Receptor-informed network control theory
links LSD and psilocybin to a flattening of the brain's control energy
landscape. \emph{Nature Communications}, \emph{13}(1), Article 1.
\url{https://doi.org/10.1038/s41467-022-33578-1}

\bibitem[\citeproctext]{ref-Sladky2011}
Sladky, R., Friston, K. J., Tröstl, J., Cunnington, R., Moser, E., \&
Windischberger, C. (2011). Slice-timing effects and their correction in
functional MRI. \emph{NeuroImage}, \emph{58}(2), 588--594.
\url{https://doi.org/10.1016/j.neuroimage.2011.06.078}

\bibitem[\citeproctext]{ref-Smith2004}
Smith, S. M., Jenkinson, M., Woolrich, M. W., Beckmann, C. F., Behrens,
T. E. J., Johansen-Berg, H., Bannister, P. R., De Luca, M., Drobnjak,
I., Flitney, D. E., Niazy, R. K., Saunders, J., Vickers, J., Zhang, Y.,
De Stefano, N., Brady, J. M., \& Matthews, P. M. (2004). Advances in
functional and structural MR image analysis and implementation as FSL.
\emph{NeuroImage}, \emph{23 Suppl 1}, S208--219.
\url{https://doi.org/10.1016/j.neuroimage.2004.07.051}

\bibitem[\citeproctext]{ref-Tagliazucchi2014}
Tagliazucchi, E., Carhart-Harris, R., Leech, R., Nutt, D., \& Chialvo,
D. R. (2014). Enhanced repertoire of brain dynamical states during the
psychedelic experience. \emph{Human Brain Mapping}, \emph{35}(11),
5442--5456. \url{https://doi.org/10.1002/hbm.22562}

\bibitem[\citeproctext]{ref-Tagliazucchi2016}
Tagliazucchi, E., Roseman, L., Kaelen, M., Orban, C., Muthukumaraswamy,
S. D., Murphy, K., Laufs, H., Leech, R., McGonigle, J., Crossley, N.,
Bullmore, E., Williams, T., Bolstridge, M., Feilding, A., Nutt, D. J.,
\& Carhart-Harris, R. (2016). Increased Global Functional Connectivity
Correlates with LSD-Induced Ego Dissolution. \emph{Current biology: CB},
\emph{26}(8), 1043--1050.
\url{https://doi.org/10.1016/j.cub.2016.02.010}

\bibitem[\citeproctext]{ref-Timmermann2023}
Timmermann, C., Roseman, L., Haridas, S., Rosas, F. E., Luan, L.,
Kettner, H., Martell, J., Erritzoe, D., Tagliazucchi, E., Pallavicini,
C., Girn, M., Alamia, A., Leech, R., Nutt, D. J., \& Carhart-Harris, R.
L. (2023). Human brain effects of DMT assessed via EEG-fMRI.
\emph{Proceedings of the National Academy of Sciences}, \emph{120}(13),
e2218949120. \url{https://doi.org/10.1073/pnas.2218949120}

\bibitem[\citeproctext]{ref-Viol2017}
Viol, A., Palhano-Fontes, F., Onias, H., De Araujo, D. B., \&
Viswanathan, G. M. (2017). Shannon entropy of brain functional complex
networks under the influence of the psychedelic Ayahuasca.
\emph{Scientific Reports}, \emph{7}(1), 7388.
\url{https://doi.org/10.1038/s41598-017-06854-0}

\bibitem[\citeproctext]{ref-Weber2010}
Weber. (2010). Htr2a gene and 5-HT2A receptor expression in the cerebral
cortex studied using genetically modified mice. \emph{Frontiers in
Neuroscience}. \url{https://doi.org/10.3389/fnins.2010.00036}

\bibitem[\citeproctext]{ref-Whitfield-Gabrieli2012}
Whitfield-Gabrieli, S., \& Nieto-Castanon, A. (2012). Conn: a functional
connectivity toolbox for correlated and anticorrelated brain networks.
\emph{Brain Connectivity}, \emph{2}(3), 125--141.
\url{https://doi.org/10.1089/brain.2012.0073}

\bibitem[\citeproctext]{ref-Whitfield-Gabrieli2011}
Whitfield-Gabrieli, S., Nieto-Castanon, A., \& Ghosh, S. (2011).
\emph{Artifact detection tools (ART)} (Release Version 7(19), 11)
{[}Computer software{]}.

\bibitem[\citeproctext]{ref-Yeo2011}
Yeo, B. T. T., Krienen, F. M., Sepulcre, J., Sabuncu, M. R., Sabuncu, M.
R., Lashkari, D., Hollinshead, M. O., Roffman, J. L., Smoller, J. W.,
Zöllei, L., Polimeni, J. R., Fischl, B., Liu, H., \& Buckner, R. L.
(2011). The organization of the human cerebral cortex estimated by
intrinsic functional connectivity. \emph{Journal of Neurophysiology},
\emph{106}(3), 1125--1165. \url{https://doi.org/10.1152/jn.00338.2011}

\end{CSLReferences}

\end{document}
