\section{Generative effective
connectivity}\label{generative-effective-connectivity}

To further characterize the unique effects of chronic and occasional use
of psychedelics and cannabis, we analyzed the contrasts between long-term users of ayahuasca and naive users of DMT, as well as chronic and occasional users of cannabis. In Figure \ref{fig:gecrender}b, it can be seen that regions influenced strongly by DMT included the left caudal middle frontal, bilateral pars triangularis and pars opercularis, and right rostral middle frontal cortex. Ayahuasca, in contrast, resulted in more connections leading to and leaving from the bilateral insula and paracentral, as well as the left isthmus cingulate and rostral anterior cingulate. For chronic and occasional use of cannabis, chronic use of cannabis was primarily characterized by the strong influence of the bilateral isthmus cingulate and precuneus, while occasional use was characterized by orchestration in the bilateral paracentral and left inferior parietal cortex. We then computed the intersection of the top
25\% regions by strength of orchestration, or total number of connections, to identify commonalities between the long-term use of ayahuasca, use of DMT in naive users, and chronic and occasional use of cannabis. Significant connectivity was primarily shared in the bilateral pre- and post-central, superior temporal, superior frontal, and insula for ayahuasca and DMT. Similarly, the chronic and occasional use of cannabis showed common orchestration by the left superior frontal. Bilateral isthmus cingulate and caudal anterior cingulate were common to both chronic and occasional use of cannabis, but not ayahuasca and DMT, highlighting the different effects of chronic use of substances.

