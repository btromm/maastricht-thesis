
    \chapter{Methods}
\section{Data acquisition and preprocessing}
\subsection{Occasional and chronic users of cannabis} 

\textbf{Ethics statement}:
This study was conducted according to the code of ethics on human
experimentation established by the declaration of Helsinki (1964) and
amended in Fortaleza (Brazil, October 2013). The study was approved by
the Academic Hospital and University's Medical Ethics Committee (Medical
ethical review board Academic Hospital Maastricht/ Maastricht
University). All participants gave written informed consent. The Dutch
Drug Enforcement Administration gave a permit for the acquisition,
storage, and administration of cannabis.

\textbf{Participants}: The data acquisition protocols were described in
detail in a previous paper \parencite{Ramaekers2022}. 43 healthy
participants with previous experience using cannabis were scanned. For
the occasional group, occasional usage was characterized as between one
time a month and three times a week. For the chronic group, chronic
usage was characterized as at least four times a week. Both were
considered only if usage was for the past year. The study was conducted
according to a double-blind, placebo-controlled, mixed cross-over design
in cannabis users (N=43). Each participant received cannabis placebo and
cannabis (300 ug/kg THC) on separate days, separated by a minimum
wash-out period of 7 days. Treatment orders were randomly assigned to
participants.

Participants received two resting state scans at 15 minutes and 36
minutes after inhalation. The current study was registered in the
Netherlands trial register (NTR4894, first date of registration
7/11/2014). Participants in the occasional group were instructed to
refrain from drug use, including cannabis (\textgreater7 days) and
alcohol (\textgreater14 hours) prior to the testing day. Participants in
the chronic group were given the same instructions, but instructed to
refrain from cannabis use only up until 24h before testing day.

\textbf{Neuroimaging acquisition for fMRI}: Participants underwent a
resting state functional MRI. Images were acquired on a MAGNETOM 7T MR
scanner. A total of 258 whole-brain EPI volumes were acquired at rest
(TR = 1400ms; TE = 21ms; flip angle = 60 degrees; oblique acquisition
orientation; interleaved slice acquisition; 72 slices; slice thickness =
1.5mm; voxel size = 1.5 x 1.5 x 1.5 mm). During scanning, participants
were shown a black cross on a white background and were instructed to
focus on the cross while attempting to clear the mind.

Results related to the cannabis data included in this manuscript come
from analyses performed using CONN (RRID:SCR\_009550) release 22.a and
SPM (RRID:SCR\_007037) release 12.7771 \parencite{Nieto-Castanon2022,Penny2007,Whitfield-Gabrieli2012}.

\textbf{Preprocessing:} Preprocessing was performed in line with \textcite{Luppi2021}. Functional and anatomical data were
preprocessed using a flexible preprocessing pipeline including
realignment with correction of susceptibility distortion interactions,
slice timing correction, outlier detection, direct segmentation and
MNI-space normalization, and smoothing \parencite{Nieto-Castanon2020}.
Functional data were realigned using SPM realign \& unwarp procedure,
where all scans were coregistered to a reference image (first scan of
the first session) using a least squares approach and a 6 parameter
(rigid body) transformation, and resampled using b-spline interpolation
to correct for motion and magnetic susceptibility interactions \parencite{Andersson2001,Friston1995}. Temporal
misalignment between different slices of the functional data (acquired
in interleaved Siemens order) was corrected following SPM slice-timing
correction (STC) procedure, using sinc temporal interpolation to
resample each slice BOLD timeseries to a common mid-acquisition time \parencite{Henson1999,Sladky2011}. Potential outlier scans were
identified using ART as acquisitions with framewise displacement above
0.5 mm or global BOLD signal changes above 3 standard deviations, and a
reference BOLD image was computed for each subject by averaging all
scans excluding outliers \parencite{Nieto-Castanon2022a,Power2014,Whitfield-Gabrieli2011}. Functional and
anatomical data were normalized into standard MNI space, segmented into
grey matter, white matter, and CSF tissue classes, and resampled to 2 mm
isotropic voxels following a direct normalization procedure using SPM
unified segmentation and normalization algorithm with the default
IXI-549 tissue probability map template \parencite{Ashburner2007,Ashburner2005,Calhoun2017,Nieto-Castanon2022a}. Last, functional data were smoothed using
spatial convolution with a Gaussian kernel of 6 mm full width half
maximum (FWHM).

\textbf{Denoising:} In addition, functional data were denoised using a
standard denoising pipeline including the regression of potential
confounding effects characterized by white matter timeseries (8 CompCor
noise components), CSF timeseries (5 CompCor noise components), motion
parameters and their first order derivatives (12 factors), outlier scans
(below 163 factors), session and task effects and their first order
derivatives (4 factors), and linear trends (2 factors) within each
functional run, followed by bandpass frequency filtering of the BOLD
timeseries between 0.008 Hz and 0.09 Hz \parencite{Friston1996,Hallquist2013,Nieto-Castanon2020,Power2014}.
CompCor noise components within white matter and CSF were estimated by
computing the average BOLD signal as well as the largest principal
components orthogonal to the BOLD average, motion parameters, and
outlier scans within each subject's eroded segmentation masks \parencite{Behzadi2007,Chai2012}. From the number of noise terms
included in this denoising strategy, the effective degrees of freedom of
the BOLD signal after denoising were estimated to range from 78.5 to
296.4 (average 195.6) across all subjects \parencite{Nieto-Castanon2022a}.

\subsection{DMT}\label{dmt}

\textbf{Ethics:} All participants provided written informed consent for
participation in the study. The study was approved by the National
Research Ethics Committee London-Brent and the Health Research
Authority, and was conducted under the guidelines of the revised
Declaration of Helsinki (2000), the International Committee on
Harmonization Good Clinical Practices guidelnes, and the National Health
Service Resaerch Governance Framework. Imperial College London sponsored
the resaerch, which was onducted under a Home Office license for
research with Schedule I drugs.

\textbf{Participants}: The data acquisition methods were described in a
previous paper, but will be briefly reported here \parencite{Timmermann2023}. The study was a single-blind, placebo-controlled,
counter-balanced design. Volunteers participated in two testing days,
separated by two weeks. Participants were tested for drugs of abuse and
involved in two separate scanning sessions on each test day. In the
initial, task-free session, participants received intravenous (IV)
administration of either placebo (10mL sterile saline) or 20mg DMT
fumarate (in 10mL sterile saline) injected over 30 seconds.
Resting-state sessions lasted 28 minutes with DMT or placebo
administered at the end of the 8th minute and scanning was over 20
minutes after injection. Participants laid in the scanner with an eye
mask on. A second session was followed with the same procedure as the
first.

\textbf{Neuroimaging Acquisition for fMRI:} Images were acquired with a
3T Siemens Magnetom Verio syngo MR B17 scanner using 12-channel head
coil for compatibility with EEG acquisition. Functional imaging was
carried out with a T2*-weighted BOLD-sensitive gradient echo planar
imaging sequence (TR = 2000 ms, TE = 30ms, TA = 28.06 ms, flip angle =
80 degrees, voxel size = 3 x 3 x 3mm, 35 slices, interslice distance =
0mm). Whole brain T1-weighted structural images were also acquired.

\textbf{Preprocessing:} Preprocessing was performed in line with
previous work \parencite{Timmermann2023}. Steps consisted of 1) despiking
(3dDespike, Analysis of Functional Neuroimages (AFNI)) \parencite{Cox1996}, 2)
slice-timing correction (3dTshift, AFNI), 3) motion correction
(3dvolreg, AFNI) by registering each volume to the most similar volume
with least squares to all others, 4) brain extraction (BET, FSL) \parencite{Smith2004}, 5) rigid body registration to anatomical scans, 6)
nonlinear registration to 2mm MNI brain (Symmetric Normalization,
Advanced Normalization Tools (ANTS)) \parencite{Avants2009}, 7) scrubbing
(FD threshold = 0.4) and scrubbed volumes were replaced with mean of
surrounding volummes. Additional preprocessing steps included the use of
8) spatial smoothing (FWHM = 6mm) (3dBlurInMask, AFNI), 9) band-pass
filtering between 0.01 and 0.08 Hz (3dFourier, AFNI), 10) linear and
quadratic detrending (3dDetrend, AFNI), 11) regressing out nine nuisance
regressors (6 motion related, 3 anatomical) with band-pass filter with
same filter as in step 9). Anatomical nuisance regressors included
ventricles (FreeSurfer, eroded in 2mm space) \parencite{Dale1999},
draining veins (FSL CSF minus FreeSurfer ventricles, eroded in 1mm
space), local white matter (FSL WM minus FreeSurfer subcortical gray
matter structures, eroded in 2mm space). Local WM regression was
performed with AFNI 3dLocalstat, calculating mean local WM timeseries
for each voxel using 25mm radius sphere centered on each voxel.
\subsection{Chronic users of ayahuasca} 
\textbf{Ethics statement:} 
The study was conducted according to the code of ethics on human experimentation established by the Declaration of Helsinki (1964) and amended in Fortaleza (Brazil, October 2013) and in accordance with the Medical Research involving Human Subjects Act (WMO) and was approved by the Academic Hospital and University's Medical Ethics committee (Medical ethical review board Academic Hospital Maastricht/Maastricht University)\\ (NL70901.068.19/METC19.050). All participants were fully informed of all
procedures, possible adverse reactions, legal rights and
responsibilities, expected benefits, and their right to voluntary
termination without consequences.

\textbf{Participants}: The data acquisition protocols were described in
detail in a previous paper \parencite{Mallaroni2022}. Twenty four
participants were enrolled in a within-subject, fixed-order
observational study. The cohort consisted of experienced members of the
Dutch chapter of the church Santo Daime. Participants underwent two
consecutive test days -- one baseline followed by another under the
influence of ayahuasca. Participants administered to themselves a volume
of ayahuasca equivalent to their usual dose (0.045 mg/kg), prepared from
a single batch by the Church of Santo Daime and analyzed according to
referencing standards (see previous paper for details). To facilitate
and naturalize the study, participants drank ayahuasca while initiating
the works in company of fellow members. Dosing schedules were stratified
across lab visits with testing performed within 4 pairs of visits, with
6 subjects per cycle. The brew used contained 0.14 mg/mL DMT, 4.50mg/ML
harmine, 0.51mg/mL harmaline, and 2.10 mg/mL tetrahydroharmine.
Ceremonies were organised and supervised by the Santo Daime church. The
research term at Maastricht University was not involved in the
organization of the rituals, production, dosing, or administration of
ayahuasca.

\textbf{Neuroimaging acquisition for fMRI:} Participants underwent
resting state functional MRI. Images were acquired on a MAGNETOM 7T MR
scanner. Participants underwent a structural MRI (60 minutes
post-treatment), single-voxel proton MRS in the PCC (70 minutes
post-treatment), visual cortex (80 minutes post-treatment), and fMRI (90
minutes post-treatment) during peak subjective effects. T1-weighted
anatomical images were acquired with magnetization-prepared 2 rapid
acquisition gradient-echo (MP2RAGE) sequence (TR = 4500ms, TE = 2.39 ms,
TI1 = 900ms, TI2 = 2750ms, flip angle 1 = 5 degrees, flip angle 2 = 3
degrees, voxel size = 0.9 mm isotropic, matrix size = 256 x 256 x 192,
phase partial Fourier = 6/8, GRAPPA = 3 with 24 reference lines,
bandwidth = 250 Hz/pixel). 500 whole brain echo planar (EPI) volumes
were acquired at rest (TR = 1400ms; TE = 21ms; field of view = 198mm;
flip angle = 60 degrees; oblique acquisition orientation; interleaved
slice acquisition; 72 slices; slice thickness = 1.5mm; voxel size = 1.5
x 1.5 x 1.5 mm) followed by 5 phase encoding volumes for EPI unwarping.
Participants were shown a black cross on a white background and were
instructed to focus on the cross during EPI acquisition.

Results related to the ayahuasca data included in this manuscript come
from analyses performed using CONN (RRID:SCR\_009550) release 21.a and
SPM (RRID:SCR\_007037) release 12.7771 \parencite{Nieto-Castanon2022,Penny2007,Whitfield-Gabrieli2012} .

\textbf{Preprocessing:} Preprocessing was performed with the same steps
as occasional and chronic users of cannabis. Denoising was performed
with the same standard denoising pipeline, including regression of
potential confounding effects characterized by white matter time-series
(5 CompCor noise components), CSF time-series (5 CompCor noise
components), motion parameters and their first and second order
derivatives (18 factors), outlier scans (below 206 factors), session and
task effects and their first order derivatives (4 factors), and linear
trends (2 factors) within each functional run, followed by bandpass
frequency filtering of the BOLD timeseries between 0.008 Hz and 0.09 Hz \parencite{Friston1996,Hallquist2013,Nieto-Castanon2020,Power2014}. CompCor noise components within white matter
and CSF were estimated by computing the average BOLD signal as well as
the largest principal components orthogonal to the BOLD average, motion
parameters, and outlier scans within each subject's eroded segmentation
masks \parencite{Behzadi2007,Chai2012}. From the number of
noise terms included in this denoising strategy, the effective degrees
of freedom of the BOLD signal after denoising were estimated to range
from 139.1 to 214 (average 202.2) across all subjects \parencite{Nieto-Castanon2022a}.
\subsection{Structural connectivity} 
To reconstruct
a structural connectivity matrix for whole-brain modelling, we obtained
multishell diffusion-weighted imaging data from 32 participants from the
HCP database (scanned approx. 89 minutes) \parencite{Kringelbach2023}.
Acquisition parameters are described in detail on the HCP website \parencite{Setsompop2013}. Diffusion tensor imaging data was parcellated
according to the aforementioned DBS80 scheme. The connectivity matrix
was weighted.
\section{Analysis}
Trophic coherence is
a general method which allows us to infer the directedness and
hierarchical levels for each brain region across the brain through the
conversion of effective weighting of the existing anatomical
connectivity as derived by a causal mechanistic whole-brain model into a
directed graph \parencite{MacKay2020, Johnson2014}. This allows for examination of changes in the functional
hierarchical organization of the brain across varying conditions. 
\subsection{Empirical functional connectivity}
The functional connectivity (FC)
$FC_{ij}^{empirical}$ is a matrix of Pearson correlations across the
fMRI BOLD timeseries activity between brain regions, \(i\) and \(j\), in
different conditions. Spatially normalized brains (MNI-space) were
parcellated into 62 cortical regions from the Mindboggle-modified
Desikan-Killiany parcellation \parencite{Desikan2006}, with the addition
of 18 subcortical regions (9 regions per hemisphere) including the
hippocampus, amygdala, subthalamic nucleus (STN), globus pallidus
internal segment (GPi), global pallidus external segment (GPe), putamen,
caudate, nucleus accumbens, and thalamus from the Gasser parcellation
(Glasser et al., 2016). This parcellation is referred to synonymously in
previous literature as the DBS80 or DK80 custom parcellation \parencite{Capouskova2022,Deco2021,Desikan2006,Gomes2020,Klein2012,Kringelbach2023}.

\subsection{Quantifying causal interactions through the level of
irreversibility} 

Previous research has shown that capturing the asymmetry
in a temporal process, the arrow of time, by comparing time-shifted
correlations between the forward and reversed BOLD fMRI time-series
provides a quantification of the level of irreversibility \& the degree
of non-equilibrium in brain dynamics, as well as the degree to which one
brain region is driving another \parencite{Deco2022,Kringelbach2023}. An increase in irreversibility is associated with increased
directionality of information flow and a resulting high level of
hierarchical reorganization. It is by this
notion that asymmetry in the directionality of information flow allows
for determination of asymmetry in both space and time, resulting in
distinct spatiotemporal hierarchies \parencite{Deco2019,Golesorkhi2021,Kobeleva2021}.

The construction of the reversed time-series is done simply by reversing
the natural forward evolution of the BOLD signal for each voxel across
the space of the brain. The causal dependency between two time-series
\(x(t)\) and \(y(t)\) is measured through the time-shifted correlation

\begin{equation}
c_{forward}(\Delta t) = <x(t),y(t+\Delta t)>
\end{equation}
 
and for the reversed time-series the time-shifted correlation is
given by \parencite{Deco2022, Kringelbach2023}: 
\begin{equation}
c_{reversed}(\Delta t) = <x^{r}(t),y^{r}(t+\Delta t)>
\end{equation}

At a given time \(\Delta t = T\), the level of irreversibility can be
computed as the absolute difference between the value of the correlation
between the forward and reversed time-series, respectively. \(\Delta t\)
represents the time shifting parameter \(T\):
\begin{equation}
I_{x,y}(T) = |c_{forward}(T) - c_{reversal}(T)|
\end{equation}

The value of \(T\) is given by, for each dataset, iterating through
values of \(T\) and identifying the value which provides the strongest
difference between conditions. Maximal differences between the
correlation of the forward and reversed time-series can be found when
one time-series has a strong dependency on time that the other does not.
The forward \(x_i(t)\) and reversal \(x^{(r)}_i(t)\) matrices are
defined as the dynamical evolution of the variable describing the
system, wherein \(i\) represents the different dimensions of the system.
The functional causal dependencies for the forward and reversed matrices
are then expressed by the normalized mutual information based on their
respective time-shifted correlations:


\begin{equation}
FS_{forward,ij}(\Delta t) = -\frac{1}{2}log[1-<x_i(t),x_j(t+\Delta t)>^2]
\end{equation}

\begin{equation}
FS_{reversal,ij}(\Delta t) = -\frac{1}{2}log[1-<x_i^{(r)}(t),x_j^{(r)}(t+\Delta t)>^2]
\end{equation}


Mutual information is utilized as the relationship between variables in
the multi-dimensional time-series is not necessarily linear. The
absolute quadratic difference of these matrices is the irreversibility
matrix


\begin{equation}
I=||FS_{forward}(T) - FS_{reversal}(T)||_2
\end{equation}


where \(I\) is the mean value of the absolute squares of the difference
between the forward and reversed matrices. If the difference matrix is
defined as such


\begin{equation}
FS_{diff,ij} = [FS_{forward,ij}(T) - FS_{reversal,ij}(T)]^2
\end{equation}
then the matrix is simply the square of the elements of the
difference between the forward and reversal matrices. The level of NR,
\(I\), is then the mean of the elements of \(FS_{diff}\). A simple
measure of hierarchy is the standard deviation of \(I\), which measures
the variability in the asymmetry of brain activity. 

\subsection{Generative effective connectivity of the arrow of time}
As previously described by \textcite{Kringelbach2023}, local dynamics of each brain region are described by
the normal form of a supercritical Hopf bifurcation. The whole-brain dynamics are given by coupling nodes through the
inclusion of a diffusive coupling term, common difference coupling,
representing the input received by a region \(n\) from every other
region \(p\), weighted by the GEC, \(G_{np}\). The dynamics of
individual nodes are given by the normal form of a supercritical Hopf
bifurcation in Cartesian coordinates with an additive Gaussian noise
\(\eta_n(t)\) and standard deviation \(\beta\):


\begin{equation}
\frac{dx_n}{dt} = [a_n - x^2_n - y^2_n]x_n - \omega_ny_n+\sum_{p=1}^N{G_{np}(x_p-x_n)} + \beta\eta_n(t)
\end{equation}

\begin{equation}
\frac{dy_n}{dt} = [a_n - x^2_n - y^2_n]y_n + \omega_nx_n+\sum_{p=1}^N{G_{np}(y_p-xy_n)} + \beta\eta_j(t)
\end{equation}


The normal form has a supercritical bifurcation \(a_n=0\), where when
\(a_n>0\), the system is engaged in a stable limit cycle with frequency
\(f_n = \omega_n/2\pi\). When \(a_n<0\), local dynamics are in a stable
fixed point which represents the low activity noise state. Intrinsic
frequencies \(\omega_n\) are estimated from the data given by the
averaged peak frequency of narrowband BOLD signals of each brain region
(n = 1, \ldots, 80). The best fit for \(a_n\) will be calculated after
model estimation. The noise factor \(\beta\) is set, by default, to
0.01.

The whole-brain model was constructed by fitting the existing anatomical
connectivity to the empirical functional connectivity (FC) matrix and
the time-delayed covariance matrix, or the covariance of
irreversibility. Effective connectivity was simulated for the ayahuasca
users, DMT users, and occasional and chronic cannabis users.
Optimization of the GEC between brain regions is performed by comparing
the output of the model with empirical measures of the forward and
reversed time-shifted correlations and the whole-brain functional
connectivity. A heuristic gradient descent algorithm is used to update
and optimize the fit of the GEC, error estimated by mean squared error
between the empirical and simulated functional connectivity matrix and
covariance of irreversibility. All values are transformed into mutual
information, assuming Gaussianity, to work only positive values:


\begin{equation}
    \begin{aligned}
        G_{ij} = G_{ij} + \epsilon(FS_{ij}^{empirical} - FS_{ij}^{model}) - \epsilon'\biggl\{ \Bigl[FS_{forward,ij}^{empirical}(T) - \\FS^{empirical}_{reversal,ij}(T)\Bigr] - \Bigl[FS_{forward,ij}^{model}(T) - FS^{model}_{reversal,ij}(T)\Bigr]\biggr\}
    \end{aligned}
\end{equation}




\(FS_{ij}\) is based on the functional connectivity matrix \(FC_{ij}\)
as mutual information obtained by:


\begin{equation}
FS_{ij}=-\frac{1}{2}log[1-(FC_{ij})^2]
\end{equation}
The model was initialized with the anatomical connectivity obtained
from probabilistic tractography from dMRI and iterated with the updated
GEC until the fit converged toward a stable value \(a_n\), with
\(\epsilon = 0.0005\) and \(\epsilon^`=0.0001\). Only known existing
connections in either hemisphere were updated, with one exception -- the
algorithm also updates homolog connections between the same regions in
either hemisphere, given that tractography is less accurate when
accounting for this connectivity. Model results are computed for each
participant, and averaged over as many simulations as there are
participants for each condition. The model was linearized with the
addition of a Jacobian matrix for steady state estimation. Updated
covariance matrices and functional connectivity matrices were calculated
by solving the Sylvester equation for the Jacobian with noise
covariances. 

\subsection{Trophic coherence} The GEC matrix represents the
effective weighting between different nodes. In other words, it provides
connection strengths as edges between brain regions. The advantage of
utilizing a whole-brain model to construct the effective connectivity of
the FC and the NR matrix is that it allows for causal inference of the
directionality of influence that one region in the brain has on another.
Trophic coherence is a property of directed graphs that has been
previously used as a statistical predictor of the linear stability of
food webs, but can be applied to any directed graph \parencite{Johnson2014, MacKay2020}. Here we use an improved notation of trophic
coherence, first introduced by \textcite{MacKay2020}, that allows for non-existence of basal nodes (a node with no
incoming edges) and the examination of reverse flows -- this is
important for brain networks, as brain regions do not simply interact in
a feed-forward manner, but have feedback loops and self-loops.

If we consider a directed network with a set \(N\) of nodes and set
\(E\) of edges, we can suppose there is at most one edge from node \(m\)
to node \(n\), with the possibility of an edge from \(n\) to \(m\) as
well. Each edge has some weight \(w_{mn}>0\). This describes the
effective weighting, or functional influence, of one brain region on
another. If an edge does not exist between a node \(m\) and \(n\), we
denote this with \(w_{mn}=0\). We then define the in-degree and
out-degree for each node \(n\) as


\begin{equation}
w_n^{in}=\sum_{m}{w_{nm}} \quad\mathrm{and}\quad w_n^{out}=\sum_{m}{w_{mn}}.
\end{equation}
 The total weight of each node \(n\) is denoted as 
\begin{equation}
u_n=w_n^{in}+w_n^{out},
\end{equation}
 which can be given alternatively by the sum of sum of rows and
columns, respectively, for a weighted matrix \(W\), where the sum of
rows represents the vectorized in-degree, and sum of columns represents
the vectorized out-degree. The imbalance for each node \(n\) is given by

\begin{equation}
v_n=w_n^{in}-w_n^{out},
\end{equation}
which can be alternatively provided by the difference between sum of
rows and sum of columns for \(W\). The weighted graph-Laplacian operator
\(\Lambda\) can be defined in matrix form by 
\begin{equation}
\Lambda = diag(u) - W - W^T,
\end{equation}
 where T is the transpose of the weighted matrix \(W\). The
hierarchical levels for each brain region can then be solved from the
solution \(\gamma\) for the linear system of equations 
\begin{equation}
\Lambda\gamma = v.
\end{equation}
To force the system to always have a unique solution, it is possible
to add an arbitrary constant to one edge of the network. Specifically,
one must add an arbitrary constant to each connected component of the
network, or the maximal subset of the network where one can move,
ignoring direction of edges, between any node \(m\) and \(n\) in the
component. Though anatomical connectivity is sparse in the brain, there
is only one connected component by definition. In our case, we used the
self-loop in the first node. Furthermore, one can normalize the
hierarchical levels, thereby also creating a basal node in the network,
by subtracting from each hierarchical level the minimum of the vector
\(\gamma\). Following this, trophic incoherence can be given by 
\begin{equation}
F_0=1-\frac{\sum_{mn}w_{mn}{(\gamma_n-\gamma_m-1)^2}}{\sum_{mn}{w_{mn}}}.
\end{equation}
The trophic coherence is then defined by \(1-F_0\). A network is
considered to be maximally coherent is \(F_0=0\) and maximally
incoherent if \(F_0=1\). Maximally coherent networks have nodes that
fall evenly onto defined trophic levels, while incoherent networks have
many nodes lying on fractional trophic levels. We can then treat trophic
coherence as the directedness of the network.

Emphasis was placed on robust connections and achieved by thresholding
weighting between regions such that only edges with \(w_{mn}>0.015\) are
considered to have an existing connection for visualization purposes.
Statistical analysis is carried out with weak edges intact as weak
connections often represent important connections between clusters.
\subsection{SVM for condition classification} 
We used a error-correcting output code support vector machine
as implemented with the MATLAB function \texttt{fitcecoc}, with 90\%
training and 10\% validation split. We optimized hyperparameters with
automatic settings and utilized \texttt{expected-improvement-plus} as an
acquisition function. The function returns a fully trained model
utilizing the predictors in the input with class labels. We then
cross-validated the model with 20 k-fold cross-validation, and evaluated
the general error and accuracy. The SVM uses inputs for each dataset and
for all datasets and conditions, to predict differences in the
distribution of hierarchical levels across baseline and acute
administration conditions, as well as between occasional and chronic
users of ayahuasca/DMT and cannabis. 
\subsection{Statistical analysis}
Changes
in irreversibility, hierarchy, and trophic coherence were calculated for
each condition with a Wilcoxon rank-sum test with \(\alpha = 0.05\) as
the threshold for statistical significance, implemented in the R package
\texttt{ggsignif} \parencite{ggsignif, Tidyverse, Package}. For DMT, we fit post-DMT data (irreversibility or
directedness) to a linear mixed effects model with the MATLAB function
\texttt{fitlme} with pre-injection, or baseline, data as a covariate and
group as the fixed effect \parencite{MATLAB}. Furthermore, we tested for interaction
effects between group and pre-injection data.

A linear mixed effects model was fit to the ayahuasca data with
ayahuasca or baseline condition as the outcome variable with time since
last ceremony (recency) and the number of total ceremonies, as well as
their interaction, as fixed effects, and max DMT concentration in the
blood (ng/ml) as a covariate. Individual subject variance at the
intercept was allowed in the model. Linear fit was determined with the
MATLAB function \texttt{polyfit} and \(R^2\) as the squared correlation
coefficient.

Changes across conditions for the full 80 region and functional network
hierarchical levels were obtained via a three-step process.
Non-parametric, two-tailed 10,000 iteration permutation testing was
performed for each region or functional network with a p-threshold of
0.01. Benjamini-Hochberg False Discovery Rate was used with a threshold
of 0.2 to identify false discoveries and correct for multiple
comparisons \parencite{Benjamini1995}. 